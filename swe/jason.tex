%%%%%%%%%%%%%%%%%
% This is an sample CV template created using altacv.cls
% (v1.6, 21 May 2021) written by LianTze Lim (liantze@gmail.com). Now compiles with pdfLaTeX, XeLaTeX and LuaLaTeX.
%
%% It may be distributed and/or modified under the
%% conditions of the LaTeX Project Public License, either version 1.3
%% of this license or (at your option) any later version.
%% The latest version of this license is in
%%    http://www.latex-project.org/lppl.txt
%% and version 1.3 or later is part of all distributions of LaTeX
%% version 2003/12/01 or later.
%%%%%%%%%%%%%%%%

\documentclass[10pt,a4paper,ragged2e,withhyper]{altacv}
%% AltaCV uses the fontawesome5 and packages.
%% See http://texdoc.net/pkg/fontawesome5 for full list of symbols.
\definecolor{primary}{HTML}{000000}
\definecolor{secondary}{HTML}{0D47A1}
\definecolor{accent}{HTML}{263238}
\definecolor{links}{HTML}{1565C0}
\definecolor{VividPurple}{HTML}{0D47A1}
\definecolor{SlateGrey}{HTML}{2E2E2E}
\definecolor{LightGrey}{HTML}{2E2E2E}
\colorlet{heading}{VividPurple}
\colorlet{tagline}{VividPurple}
\colorlet{accent}{VividPurple}
\colorlet{emphasis}{SlateGrey}
\colorlet{body}{LightGrey}

% Change the page layout if you need to
\geometry{left=1.25cm,right=1.25cm,top=1.5cm,bottom=1.5cm,columnsep=1.2cm}

% The paracol package lets you typeset columns of text in parallel
\usepackage{paracol}

% Change the font if you want to, depending on whether
% you're using pdflatex or xelatex/lualatex
\ifxetexorluatex
  % If using xelatex or lualatex:
  \setmainfont{Roboto Slab}
  \setsansfont{Lato}
  \renewcommand{\familydefault}{\sfdefault}
\else
  % If using pdflatex:
  \usepackage[rm]{roboto}
  \usepackage[defaultsans]{lato}
  % \usepackage{sourcesanspro}
  \renewcommand{\familydefault}{\sfdefault}
\fi

% Change the colours if you want to
%\definecolor{SlateGrey}{HTML}{2E2E2E}
%\definecolor{LightGrey}{HTML}{666666}
%\definecolor{DarkPastelRed}{HTML}{450808}
%\definecolor{PastelRed}{HTML}{8F0D0D}
%\definecolor{GoldenEarth}{HTML}{E7D192}
%\colorlet{name}{black}
%\colorlet{tagline}{PastelRed}
%\colorlet{heading}{DarkPastelRed}
%\colorlet{headingrule}{GoldenEarth}
%\colorlet{subheading}{PastelRed}
%\colorlet{accent}{PastelRed}
%\colorlet{emphasis}{SlateGrey}
%\colorlet{body}{LightGrey}

% Change some fonts, if necessary
\renewcommand{\namefont}{\Huge\rmfamily\bfseries}
\renewcommand{\personalinfofont}{\footnotesize}
\renewcommand{\cvsectionfont}{\LARGE\rmfamily\bfseries}
\renewcommand{\cvsubsectionfont}{\large\bfseries}


% Change the bullets for itemize and rating marker
% for \cvskill if you want to
\renewcommand{\itemmarker}{{\small\textbullet}}
\renewcommand{\ratingmarker}{\faCircle}

\begin{document}
\name{Jason Ernst}
\tagline{Senior Software Engineer}

\personalinfo{%
  \github{github.com/compscidr}
  \linkedin{linkedin.com/in/jasonbernst/}
  \twitter{@compscidr}
  \homepage{jasonernst.com}
  
  \email{ernstjason1@gmail.com}
  \phone{+16507401486}
  \mailaddress{607 Canoe Court, 94065}
  \location{Redwood City, CA, United States}
}

\makecvheader

\cvsection {Professional Summary}
Experienced, distributed systems software architect \& developer. Passionate about delivering test-driven, customer-focused, high-performance, software systems. Skilled at communicating technical vision to a variety of stakeholders including executive teams, product, and business units, funding the vision, and growing the team to deliver a product.

%% Depending on your tastes, you may want to make fonts of itemize environments slightly smaller
\AtBeginEnvironment{itemize}{\small}

%% Set the left/right column width ratio to 6:4.
\columnratio{0.6}

% Start a 2-column paracol. Both the left and right columns will automatically
% break across pages if things get too long.
\begin{paracol}{2}

\cvsection{Experience}
\cvevent{Principal Software Engineer}{ Peep Networks Inc. }{June 2022-Present} {Boston, United States}
\begin{itemize}
\item Leading the technical development of mobile mesh network software for Android, iOS and open hardware routers.
\end{itemize}

\cvevent{Senior Software Engineer}{ \href{https://vimeo.com/user106461545/videos}{Rapid Robotics Inc.} }{Aug 2020-June 2022} {San Francisco, United States}
\begin{itemize}
\item Significant contributions to architecture of the c++ motion stack, path planning \& parameterization which is core to movement of the robot. 
\item Designed and Implemented a backend python service to re-align waypoints in a task so the robot can be moved in an out of a workspace without reprogramming the waypoint positions using fiducial markers and cameras
\item Improved performance and memory usage with heaptrack, perf, gdb and other runtime debugging, profiling and analysis tools. Contributed to the design and implementation of network protocols between services, and devices like cameras, sensors and robots. Integrated opentelemetry into our motion stack for realtime health and performance metrics feeding into grafana.
\item Designed \& implemented various platform services in c++ and python including asset storage \& retrieval, neural network inference services, and a simulation backend which exposed the motion stack along with a front-end javafx 3d renderer which allows for previewing waypoints, range of motion of the arm, and  collisions within the environment.
\end{itemize}

\cvevent{Senior Software Engineer}{ \href{https://www.youtube.com/watch?v=7_L0wiW9MB0}{Osaro Inc.} }{Sept 2019-Aug 2020} {San Francisco, United States}
\begin{itemize}
\item Implemented event-driven inter-process/thread communication in python using ZMQ improving efficiency and performance
\item Reduced delay in acquiring images in python from network connected Intel RealSense cameras from 8+ seconds to 0.05s
\item Designed a distributed statsd / carbon graphite architecture for a metrics collection system to replicate metrics in a scalable manner to a globally available service
\item Developed devops infrastructure in Saltstack and Terraform to deploy software and updates to edge devices on customer sites
\end{itemize}

\cvevent{CTO / Software Architect }{\href{https://www.youtube.com/playlist?list=PL1HQc9Sqe59hFi9VfWAT68su4nLdH2DOj}{RightMesh}}{2016 -- 2019}{Vancouver, Canada}
\begin{itemize}
\item Developed wireless protocols on Android and Linux in java which achieved 30Mbps over a 10+ hops p2p user-space mesh network.
\item Drove initiative towards a dependency-injected layered architecture providing mux/demux-ing, routing, transport, encryption and decentralized payment layers with ethereum smart contracts.
\item Improved the reliability of the codebase by spearheading unit tests, integration tests, system tests, instrumented tests along with tools like jacoco, checkstyle and gitlab-ci to ensure full-code coverage was required before branches were merged.
\item Fundraising driven by my technical vision raised \$32M through an Initial Coin Offering (ICO), MITACs, SRED and NSERC, which grew technical team from myself to 15 local developers, and 40 remote developers across 4 offices of which I was the lead as CTO.
\end{itemize}

\cvevent{CTO / Software Architect}{ \href{https://www.youtube.com/playlist?list=PL1HQc9Sqe59hfpogQN5kNTqNuThDw2oZ_}{Redtree Robotics} }{2014 -- 2016} {Guelph, Canada}
\begin{itemize}
\item Developed c/c++ middleware, custom Linux kernel \& driver for IEEE 802.11 mesh networking, Bluetooth and 4G for our plug-and-play swarm robotics hardware platform. Managed a small team of devs.
\end{itemize}

\cvevent{Web Developer}{ Conestoga College }{2008-2009} {Kitchener, Canada}
\begin{itemize}
\item Developed and maintained school websites using java, tomcat and oracle db. Hired co-op and mentored co-op students.
\end{itemize}


\cvsection{Technical Skills}
\cvskill{Java, Android, Jacoco, Dep Injection \& Mocking}{5}
\cvskill{CI/CD, Docker, Checkstyle, Lombok, Ansible}{5}
\cvskill{Protocol Design \& Implementation, Concurrency}{5}
\cvskill{RxJava, C/C++, Makefile, Shell Scripting, PHP}{4}
\cvskill{Javascript, SQL, Schema Design, Python, Kotlin}{3}
\cvskill{OpenCV, Tensorflow, Terraform, K8s, Go, Rust}{2}

\divider

\switchcolumn

\cvsection{Education}
\cvevent{Ph.D. Computer Science}{}{2009 -- 2015}{Guelph, Canada}
\begin{itemize}
    \item Distributed Systems: Wireless Mesh Networks
\end{itemize}
\smallskip

\cvsection{Invited Talks}
\smallskip
\begin{itemize}
    \item Standing Committee on Industry, Science \& Technology, Government of Canada, 2019.
    \item MNM Workshop, TMA Conference 2019: Keynote
    \item Pint of Science 2019: Invited Speaker
    \item AIDecentralized 2018: Keynote and Panelist
\end{itemize}

\cvsection{Research \& Patents}
\smallskip
\begin{itemize}
\item W. Cai, Z. Wang, J.B Ernst, Z. Hong, C Feng, VCM Leung. Decentralized applications: The blockchain-empowered software system, IEEE Access 6. 2018.
\item Jason B Ernst. Method for establishing network clusters between networked devices. Provisional(US 62343056), 2016.
\item Jason B Ernst, Stefan C Kremer, and Joel JPC Rodrigues. Heterogeneous wireless network RAT selection with multiple operators and service contracts. in IEEE International Conference on Communications (ICC), pp. 6011–6017. 2015.
\item Jason B Ernst, Stefan C Kremer, and Joel JPC Rodrigues. A survey of QOS/QOE mechanisms in heterogeneous wireless networks. Physical Communication, 13:61–72, 2014.
%\smallskip
\end{itemize}


\cvsection{Awards}
\cvachievement{\faTrophy}{Young Alumnus Award, 2019}{Presented to an individual who has distinguished himself in the profession, the community and the university.}

\smallskip \divider \smallskip

\cvachievement{\faTrophy}{MITACs Cluster Grant, 2018}{Awarded \$2.13M MITACs Cluster Grant to support 14 PhD students and 4 PDFs}

\newpage

\cvsection{Volunteer}

\cvevent{Future of Computing Academy}{Association of Computing Machinery}{}{}
\begin{itemize}
\item Co-authored \href{https://acm-fca.org/2018/03/29/negativeimpacts/}{Negative Impacts} paper, which recieved coverage in New York Times.
    \item Served on executive board shaping strategy, recruitment and as a liason between 46 members and broader ACM.
    \item Attended ACM Turing Awards in 2017, ACM Committees in 2018 and 2019.
\end{itemize}

\divider

\end{paracol}


\end{document}
